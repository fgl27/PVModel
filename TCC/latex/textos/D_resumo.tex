%Este texto está parecendo uma referência bibliográfica do teu material e não o resumo.   
%O resumo (que será o  abstract ao ser colocado em inglês) deve mostrar o problema que foi atacado, a abordagem que foi dada ao problema, seu ineditismo (se for o caso) e os resultados principais alcançados.    

\begin{abstract}

A geração e o consumo de energia são fatores cruciais para o desenvolvimento da humanidade. No Brasil, ainda há uma grande defasagem, em relação aos países mais desenvolvidos, na utilização de veículos elétricos, sendo que, provavelmente,  um dos motivos mais importantes para isso seja a falta de uma rede de recarga para veículos elétricos.  Este trabalho tem como objetivo desenvolver um sistema que pode ser usado para demonstrar que estações de recarga veiculares são financeira e energeticamente viáveis, quando são associadas à cogeração de energia, oriunda de uma fonte limpa e renovável, como a energia solar. Para isso, foram desenvolvidos modelos matemáticos e um aplicativo computacional que, por meio desses modelos, calcula e apresenta os resultados de viabilidade técnica e financeira. Os resultados obtidos mostraram que quando associamos uma estação de recarga veicular à cogeração de energia solar, o sistema se torna extremamente viável, tanto sob o ponto de vista energético, quanto sob o ponto de vista financeiro. O sistema é capaz de produzir a sua própria energia, ou seja, se torna independente da rede elétrica e se paga em alguns anos, podendo gerar, a partir desse período, um ótimo retorno financeiro e um excelente retorno social e ambiental, por meio da utilização de energia a partir de uma fonte limpa.

\end{abstract}



