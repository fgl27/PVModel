\begin{englishabstract}
 %Titulo e inglês
{A computer system capable of determining the solar energy potential of an establishment or residence}
 %Palavras chaves em inglês
{Solar energy, Electric vehicle, Vehicle charging station, Photovoltaic systems}

The generation and consumption of energy are crucial factors for the development of humanity. In Brazil, there is still a large gap, in relation to more developed countries, in the use of electric vehicles, and probably one of the most important reasons for this is the lack of a charging network for electric vehicles. This work aims to develop a system that can be used to demonstrate that vehicular charging stations are financially and energetically viable, when they are associated with cogeneration of energy from a clean and renewable source, such as solar energy. For this, mathematical models and a computational application were developed that, through these models, calculate and present the results of technical and financial feasibility. The results obtained showed that when we associate a vehicular charging station with cogeneration from solar energy, the system becomes extremely viable, both from the energy point of view, as from the financial point of view. The system is capable of producing its own energy, that is, it becomes independent from the electricity grid and pays for itself in a few years, and from that it can generate a great financial return and an excellent social and environmental return, through use of energy from a clean source.

\end{englishabstract}

