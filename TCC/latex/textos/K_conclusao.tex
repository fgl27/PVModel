\chapter{Conclusão}\label{conclusoes}

Neste trabalho foi desenvolvido um aplicativo computacional que permite obter, de forma simples, a capacidade de produção de energia, os custos e retornos financeiros de uma estação de recarga para veículos elétricos que utiliza cogeração de energia com a rede comercial de distribuição, a partir da conversão fotovoltaica.

Na revisão bibliográfica, foi apresentado o processo de produção de energia elétrica a partir da energia solar, por meio da utilização de painéis solares. Foram descritos os diferentes dispositivos envolvidos nesse processo até a entrega da energia aos equipamentos elétricos, tendo sido apresentados todos os fatores que influenciam a operação e o desempenho desses dispositivos.
O panorama do consumo e geração de energia elétrica nacional foi apresentado, mostrando a grande dependência que há, na geração de energia, em relação à fonte hídrica. Isso demonstra que há muito espaço para o crescimento da participação de outras fontes de energia elétrica, principalmente as fontes renováveis, como a energia solar fotovoltaica, pois a demanda é crescente e as fontes hídricas são limitadas.
Como o preço da energia elétrica no Brasil vem aumentando muito nos últimos anos, a produção própria de energia se torna cada vez mais atrativa. Foi utilizada a modelagem de sistema de energia fotovoltaica publicada pelo \cite{sandia} como base para o desenvolvimento do modelo deste trabalho, também foi apresentado os custos e autonomia de veículos elétricos em conjunto do sistema de recarga e estações de recarga veicular.

Utilizando todo material produzido na revisão bibliográfica foi então desenvolvido o modelo matemático capaz de apresentar os resultados propostos, foi desenvolvido um aplicativo, utilizando as linguagens Python, JavaScript, HTML e CSS capaz de ser usado para aplicar o modelo.

Com os resultados obtidos, através do aplicativo, foi possível determinar todos os custos e retornos, tanto financeiro quanto energético, do sistema proposto e assim determinar a viabilidade de implantação de estações de recarga veicular e sistema de produção de energia elétrica usando painéis fotovoltaicos. Como foi possível constatar, mesmo
com custos financeiros iniciais altos, o sistema se paga. No caso de um sistema único, o tempo que leva para o sistema amortizar o custo inicial é mais elevado quando comparamos ao sistema combinado. O sistema computacional permite que as entradas sejam alteradas de diversas maneiras a fim de simular um sistema adequado às necessidades do usuário.

Dessa forma, o objetivo de desenvolver um sistema computacional, capaz de demonstrar o potencial energético de um estabelecimento, localidade ou região, para a instalação de pontos de recarga de veículos elétricos, assim como a sua viabilidade econômica, foi cumprido.

A produção e consumo de energia, de forma consciente e renovável, vem se tornado cada vez mais importante na vida do ser humano, devido a isto os assuntos abordados neste trabalho assumem especial relevância. Portanto, esse trabalho pode ser utilizado como base, para trabalhos futuros, que visem incentivar o uso de energia limpa e renovável, através de novas soluções tecnológicas.

Em relação à questão profissional, o aplicativo desenvolvido pode ser utilizado como ponto de partida para a determinação da viabilidade técnica e econômica da produção ou coprodução de energia elétrica, neste caso a partir da energia fotovoltaica. Portanto, o impacto futuro deste trabalho dependerá muito da forma como o seu potencial for aproveitado.
