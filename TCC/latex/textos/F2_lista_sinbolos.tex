\begin{listofsymbols}{SPMD}

 \begin{longtable}[c]{ >{\centering\arraybackslash} m{2cm} >{\centering\arraybackslash} m{10cm} >{\centering\arraybackslash} m{2cm} }
    \caption{Lista de símbolos}
    \hline
    Símbolo & Descrição & Unidade \\ \hline %Primeira e ultima linha adiciona \hline apos \\
    
    $A$ & {\textbf{Ampere} \textit{é a unidade de medida da corrente elétrica no Sistema Internacional de Unidades.}} & -  \\
    
    $Albedo$ & {\textbf{$Albedo$} \textit{é a fração da irradiação horizontal global refletida. Quando a superfície é muito escura $Albedo \approx 0$ e quando a superfície é branca brilhante ou metálica $Albedo \approx 1$.}} & - \\
      
    $AOI$ & {\textbf{Ângulo de incidência} \textit{é o angulo entre os raios do Sol e o arranjo fotovoltaico.}} & ° \\
    
    $CA$ & {\textbf{A corrente alternada (CA ou AC - do inglês alternating current)} \textit{é uma corrente elétrica cujo sentido varia no tempo.}} & A  \\
    
    $CC$ & {\textbf{Corrente contínua (CC ou DC do inglês direct current)}} & A \\
    
    $C_c$ & {\textbf{Custo do valor de compra do kWh da concessionária}} & $\frac{R\$}{kWh}$ \\
    
    $C_E$ & {\textbf{Custo total das estações de recarga}} & $R\$$ \\
    
    $C_{ev}$ & {\textbf{Consumo originado das estações de recarga}} & $kWh$ \\
     
    $C_F$ & {\textbf{Custo fixo para cada 1000 W a inverter}} & $R\$$ \\
    
    $C_{pv}$ & {\textbf{Custo total do sistema PV}} & $R\$$ \\
    
    $C_{R}$ & {\textbf{Consumo originado da residencia ou estabelecimento}} & $kWh$ \\
    
    $C_s$ & {\textbf{Custo total do sistema}} & $R\$$ \\
     
    $C_T$ & {\textbf{Consumo total anual de energia}} & $\frac{R\$}{kWh}$ \\
    
    $C_u$ & {\textbf{Custo unitário de cada estação de recarga }} & $R\$$ \\
     
    $C_v$ & {\textbf{Custo do valor de venda do kWh}} & $\frac{R\$}{kWh}$ \\
     
    °C & {\textbf{Grau Celsius}} & -  \\
    
    $DHI$ & {\textbf{Irradiância horizontal difusa} \textit{ é a irradiância terrestre recebida por uma superfície horizontal que foi espalhada ou difundida pela atmosfera.}} & $\frac{W}{m^2}$  \\
    
    $DNI$ & {\textbf{Irradiância normal direta} \textit{é medida diretamente por meio de um radiômetro de cavidade absoluta.}} & $\frac{W}{m^2}$  \\
    
    $E_{b}$ & {\textbf{Componente de irradiância do feixe POA $E_b$} \textit{é calculada ajustando a irradiância normal direta (DNI) pelo ângulo de incidência (AOI).}} & $\frac{W}{m^2}$  \\
    
    $E_c$ & {\textbf{Custo total com estruturas de suporte}} & $R\$$ \\
       
    $E_{d}$ & {\textbf{Irradiância difusa do céu} \textit{é a radiação difusa na cúpula do céu.}} & $\frac{W}{m^2}$  \\
    
    $E_{g}$ & {\textbf{Irradiância em uma superfície inclinada que é refletida do solo}} & $\frac{W}{m^2}$  \\
    
    $E_i$ & {\textbf{Custo individua de cada estrutura de suporte}} & $R\$$ \\
    
     $E_{pg}$ & {\textbf{Custo da energia paga da concessionária}} & $R\$$ \\
     
    $E_R$ & {\textbf{Valor recebido da concessionária}} & $R\$$ \\

    $E_{POA}$ & {\textbf{Irradiância incidente no plano da matriz} \textit{é a irradiação total que incide na matriz de painéis.}} & $\frac{W}{m^2}$  \\
    
    $G_A$ & {\textbf{Déficit ou excedente energia, Déficit quando o resultado é negativo, excedente quando positivo}} & $kWh$ \\
    
    $GHI$ & {\textbf{{Irradiância horizontal global} \textit{é a quantidade de irradiância terrestre que cai em uma superfície horizontal à superfície da terra.}}} & $\frac{W}{m^2}$  \\

    $h_u$ & {\textbf{Horas de uso por dia da estação de recarga}} & - \\
 
    $I_c$ & {\textbf{C Custo total com inversores CC-CA}} & $R\$$ \\
 
    $I_{sc}$ & {\textbf{Short Circuit Current  (Corrente de curto-circuito)} \textit{é a correte que o painel produz quando seus terminais são curto circuitados.}} & A  \\
    
    $kW$ & {\textbf{Quilowatt ou kilowatt}}  & - \\
    
    $km$ & {\textbf{quilômetro ou quilómetro } } & - \\
    
    $m$ & {\textbf{Metro} } & -  \\
    
    $m^2$ & {\textbf{Metro quadrado} \textit{é a unidade padrão de área}} & -  \\
    
    $\Omega$ & {\textbf{Ohm}} & -  \\
    
    $P$ & {\textbf{Potência} \textit{na física, potência é a grandeza que determina a quantidade de energia concedida por uma fonte a cada unidade de tempo.}} & $\frac{W}{t}$  \\
    
    $P_E$ & {\textbf{Potência nominal da estação}} & $W$ \\
    
    $P_{EV}$ & {\textbf{Prazo de retorno de investimento estações de recarga }} & $t$ \\
     
    $P_{max}$ & {\textbf{Potência nominal máxima} \textit{é a potência nominal máxima do painel fotovoltaico, também descrita pelo simbolo $P_{mpp}$}} & $Wp$  \\
    
    $P_{PV}$ & {\textbf{Prazo de retorno de investimento sistema PV }} & $t$ \\
    
    $Q_E$ & {\textbf{Quantidade de estações}} & - \\
    
    $Q_{es}$ & {\textbf{Quantidade de estruturas}} & - \\

    $Q_D$ & {\textbf{Quantidade de dias estação de recarga em uso}} & - \\
    
    $R_E$ & {\textbf{Faturamento das estações de recarga}} & $R\$$ \\
    
    $R_T$ & {\textbf{Faturamento total do ano}} & $R\$$ \\
    
    $t$ & {\textbf{Tempo}} & -  \\
    
    $Ta$ & {\textbf{Temperatura ambiente} \textit{é temperatura medida na superfície da terra.}} & °C  \\

    $Tc$ & {\textbf{Temperatura da célula} \textit{é temperatura medida na célula fotovoltaica.}} & °C  \\
    
    $Tm$ & {\textbf{Temperatura do modulo} \textit{é temperatura medida no modulo fotovoltaico.}} & °C  \\

    $V$ & {\textbf{Volt}} & -  \\
    
    $V_{oc}$ & {\textbf{Tensão de circuito aberto} \textit{é a tensão máxima disponível em uma célula solar ocorre quando a corrente é zero.}} & $V$  \\
    
    $V_{f}$ & {\textbf{Valor fixo cobrado pela concessionária para estar conectado a rede elétrica em}} & $R\$$  \\
    
    $W$ & {\textbf{Watt} } & -  \\
    
    $W_A$ & {\textbf{Quantidade de kWh produzido em um ano pelo sistema PV} } & $\frac{kWh}{ano}$  \\
    
    $W_h$ & {\textbf{Watt-hora} \textit{é a medida de energia usualmente utilizada em eletrotécnica.}} & - \\
    
    $W_p$ & {\textbf{Watt-pico} \textit{é a unidade criada para caracterizar a potencia de pico dos painéis fotovoltaicos.}} & $W$  \\
    
    $W_{pC}$ & {\textbf{C Custo fixo por W}} & $R\$$  \\
    
    $W_{pT}$ & {\textbf{Watts Pico total do sistema PV}} & $W_p$  \\

    $W_{T}$ & {\textbf{Produção total de energia pelo sistema PV}} & $kWh$  \\

    $WS$ & {\textbf{Velocidade do vento} \textit{é a unidade criada para caracterizar a velocidade do vento sobre o painel fotovoltaico.}} & $\frac{m}{s}$  \\

    $\theta_{T}$ & {\textbf{Ângulo de inclinação da superfície } \textit{é o angulo que o painel fotovoltaico forma com a superfície.}} & °  \\
    
    $\theta_{z}$ & {\textbf{Ângulo zênite} \textit{e ângulo zênite ente o sol e terra.}} & °  \\

 \end{longtable}

\end{listofsymbols}