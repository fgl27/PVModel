\chapter{Introdução}

\section{Motivação}\label{Motivação}

O crescimento do uso de veículos elétricos, e os evidentes benefícios que isso traz para o meio ambiente e para o uso sustentável dos recursos energéticos, apresenta alguns desafios tecnológicos. Provavelmente, um dos maiores desafios para que a propulsão de veículos, por meio de motores elétricos, se torne realmente importante, sob o ponto de vista ambiental, seja a instalação de uma rede de recarga acessível e eficiente.
A instalação de uma rede de recarga, que dê o suporte adequado a esse crescimento, apresenta, por sua vez, uma série de dificuldades, tanto do ponto de vista técnico quanto do ponto de vista econômico. Um dos aspectos mais importantes, sob o ponto de vista econômico, para a viabilização dessa rede, diz respeito à possibilidade do retorno que tal empreendimento poderá gerar a empreendedores individuais.  Para que um empreendimento dessa natureza se torne viável, e gere um retorno financeiro atraente, é necessário que sua localização seja escolhida de forma a garantir um número de recargas diário economicamente compensador. Outro aspecto importante, é a possibilidade de cogeração de energia, por meio do aproveitamento da energia fotovoltaica, com o objetivo de aliviar o sistema elétrico e de aumentar a rentabilidade do investimento. Em grandes centros urbanos, encontramos facilmente áreas como shopping centers e supermercados, com amplas áreas de estacionamento a céu aberto, assim como paradouros em estradas, que podem ser explorados como estações de recarga e de geração de energia fotovoltaica. A possibilidade de se utilizar energia solar como fonte barata e sustentável de energia, pode ser um grande fator de viabilização desse tipo de estação, pois após um certo tempo é possível obter um retorno financeiro mesmo que não se tenha um fluxo adequado de veículos elétricos utilizando o sistema. Um estabelecimento dessa natureza pode se tornar economicamente viável, tanto pelo retorno financeiro gerado pelo sistema de recarga, quanto pelo sistema de cogeração de energia. Portanto, é muito provável que esta seja uma das soluções, a médio prazo, para o problema da recarga da frota de veículos elétricos. Dessa forma, a avaliação do potencial econômico, de sistemas dessa natureza, se torna fundamental, pois os investimentos só se tornarão realidade se houver uma perspectiva razoável de retorno financeiro. Tendo em vista essa perspectiva é que esse trabalho foi idealizado, apresentando como objetivo central o desenvolvimento de um sistema computacional capaz de demostrar o potencial energético de um estabelecimento, localidade ou região, para a instalação de pontos de recarga de veículos elétricos, assim como a sua viabilidade econômica.

\section{Objetivo}\label{Objetivo}

Este estudo tem os seguintes objetivos:
\begin{itemize}

    \item Desenvolver um modelo matemático capaz de determinar custos e retornos assim como a capacidade de produção de energia elétrica de um sistema fotovoltaico, utilizado como cogerador de energia para uma estação de recarga veicular;
    
    \item Desenvolver e publicar um aplicativo acessível através da internet capaz de ser usado para demonstrar os resultados do modelo desenvolvido;
    
    \item Utilizar o aplicativo para demonstrar os resultados relativos à produção de energia, custos e retornos financeiros e aos benefícios do sistema.
  
\end{itemize}